\documentclass{report}

\input{preamble}
\input{macros}
\input{letterfonts}

\title{\Huge{Theory of Linear Algebra}\\Math 2R03}
\author{\huge{Andrew Dmitri Bovbel}}
\date{Winter 2024}

\begin{document}

\maketitle
\newpage% or \cleardoublepage
% \pdfbookmark[<level>]{<title>}{<dest>}
\pdfbookmark[section]{\contentsname}{toc}
\tableofcontents
\pagebreak



\chapter{Vector Spaces}
\section{Complex Numbers}
\dfn{Complex Numbers, $\bbC$}{$\bullet$ A complex number is an ordered pair $(a,b)$ where $a,b \in$ $\bbR$ expressed in the form $a + bi$
\newline
$\bullet$ Set of all complex numbers is $\bbC$, denoted by $\bbC = \{ a + bi : a,b \in \bbR \}$ }
$\newline$

If $a \in \bbR $, we identify $a + 0i$. It is important to note that $\bbR \subset \bbC \newline$
% $\bulle t${A complex number $\bbC$ is an ordered pair (a,b) where a,b $\exists$ $\bbR$}

$i$ is simply $\sqrt{-1}$, and $i^2 = -1$

$\newline$

\dfn{Properties of complex arithmetic}{\textbf{commutativity} \newline \textbf{associativity} \newline \textbf{identities} \newline \textbf{additive inverse} \newline \textbf{multiplicative inverse} \newline \textbf{distributive property}}
% $i$ is simply $\root$, and by implication 

\end{document}